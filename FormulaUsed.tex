\documentclass[11pt,aj4]{jarticle}
\usepackage{amsmath}
\begin{document}

\title{物理エンジンをつくりたい}
\author{Fidio-lp2}
\date{2020/2/22}
\maketitle

%
\newpage
% 

\part{はじめに}

\section{この文書を作った目的}

この文書は、著者がtexの練習がてら作ったものですし、そもそも私は\\
物理や数学を苦手としている人間なので内容の正確さには欠けます.ご了承ください.

\section{この文書の構成内容}

まず、剛体シミュレーションでは主に5つの過程を経て演算結果が目に見えるように反映される。\\
この5つの処理は、順番に、

\begin{enumerate}
\item 剛体への外力の入力
\item ブロードフェーズ(衝突の検出その1)
\item ナローフェーズ(衝突の検出その2)
\item 拘束の解消
\item 剛体の位置・速度の更新
\end{enumerate}

となっている。この5つのフェーズで使用されている式について見ていく.

\part{使用した式など}

\section{剛体への外力の入力}

まだ

\section{ブロードフェーズ}

まだ

\section{ナローフェーズ}

まだ

\section{拘束の解消}

まだ

\section{剛体の位置・速度の更新}

・加速度一定の物体の位置

\[
    y
    = v_0 t + \frac{1}{2} a t^2
\]

・加速度一定の物体の速度

\[
    v
    = v_0 + a t
\]

・円筒状の慣性モーメント

\[
  \frac{1}{2} M ((R_1)^2 + (R_2)^2)
\]

\section{未分類}

    ・三次元ベクトルの性質

    \[
        \cos^2\phi_x + \cos^2\phi_y + \cos^2\phi_z = 1
    \]

    ・各座標軸のベクトル

    \[
        \left\{
        v_x = |V| \cos\phi_x
        v_y = |V| \cos\phi_y
        v_z = |V| \cos\phi_z
        \right\}
    \]

    ・外積の式

    $\vec{a}$と$\vec{b}$がなす角を$\phi$とし、\\
    $\vec{a} = (a_1, a_2, a_3)$,$\vec{b} = (b_1, b_2, b_3)$とすると、

    \[
        \begin{split}
        \vec{a} \times \vec{b}
        &=& (a_2 b_3 - a_3 b_2, a_3 b_1 - a_1 b_3, a_1 b_2 - a_2 b_1) \\

        |\vec{a} \times \vec{b}|
        &=& |\vec{a}||\vec{b}| \sin \phi
            \end{split}
    \]

\end{document}